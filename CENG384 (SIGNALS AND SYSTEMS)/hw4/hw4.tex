\documentclass[10pt,a4paper, margin=1in]{article}
\usepackage{fullpage}
\usepackage{amsfonts, amsmath, pifont}
\usepackage{amsthm}
\usepackage{graphicx}
\usepackage{float}
\usepackage{tkz-euclide}
\usepackage{listings}
\usepackage{tikz}
\usetikzlibrary{positioning,arrows}
\usepackage{pgfplots}
\pgfplotsset{compat=1.13}
\usepackage{xcolor}

\definecolor{codegreen}{rgb}{0,0.6,0}
\definecolor{codegray}{rgb}{0.5,0.5,0.5}
\definecolor{codepurple}{rgb}{0.58,0,0.82}
\definecolor{backcolour}{rgb}{0.95,0.95,0.92}

\lstdefinestyle{mystyle}{
    backgroundcolor=\color{backcolour},   
    commentstyle=\color{codegreen},
    keywordstyle=\color{magenta},
    numberstyle=\tiny\color{codegray},
    stringstyle=\color{codepurple},
    basicstyle=\ttfamily\footnotesize,
    breakatwhitespace=false,         
    breaklines=true,                 
    captionpos=b,                    
    keepspaces=true,                 
    numbers=none,                    
    numbersep=5pt,                  
    showspaces=false,                
    showstringspaces=false,
    showtabs=false,                  
    tabsize=2
}

\lstset{style=mystyle}


\usepackage{geometry}
 \geometry{
 a4paper,
 total={210mm,297mm},
 left=10mm,
 right=10mm,
 top=10mm,
 bottom=10mm,
 }
 % Write both of your names here. Fill exxxxxxx with your ceng mail address.
 \author{
  Meyvecioğlu, Hasan Ege\\
  \texttt{e2449783@ceng.metu.edu.tr}
  \and
  Şanlı, Enes\\
  \texttt{e2375749@ceng.metu.edu.tr}
}

\title{CENG 384 - Signals and Systems for Computer Engineers \\
Spring 2023 \\
Homework 4}
\begin{document}
\maketitle



\noindent\rule{19cm}{1.2pt}

\begin{enumerate}

\item %write the solution of q1
	\begin{enumerate}   
    % Write your solutions in the following items.
    \item %write the solution of q1a
   \begin{equation}
        H(jw) = \frac{Y(jw)}{X(jw)}
   \end{equation}
   \begin{equation}
       (jw+1)Y(jw) = (jw-1)X(jw)
   \end{equation}
   \begin{equation}
       y'(n) + y(n) = x'(n) - x(n)
   \end{equation}
    \item %write the solution of q1b
    \begin{equation}
        H(jw) = \frac{jw - 1}{jw +1}        
    \end{equation}
    \begin{equation}
        H(jw) = 1 -\frac{2}{jw + 1}
    \end{equation}
    \begin{equation}
        h(t) = F^{-1}\{H(jw)\} = F^{-1}\{ 1 -\frac{2}{jw + 1}\} 
    \end{equation}
    \begin{equation}
        = \delta(t) - 2e^{-t}u(t)
    \end{equation}
	\item %write the solution of q1c
    \begin{equation}
        x(t)*h(t) = y(t)
    \end{equation}
    so
    \begin{equation}
        X(jw)H(jw) = Y(jw)
    \end{equation}
    \begin{equation}
        \frac{1}{jw+2}\frac{jw - 1}{ jw +1} = Y(jw)
    \end{equation}
    \begin{equation}
        \frac{jw -1}{(jw+2)(jw+1)} = \frac{A}{jw+1} + \frac{B}{jw+2}
    \end{equation}
    \begin{equation}
        2A+B=-1, A+B = 1
    \end{equation}
    \begin{equation}
        A = -2, B=3 
    \end{equation}
    \begin{equation}
        Y(jw) = \frac{-2}{jw+1} + \frac{3}{jw+2}
    \end{equation}
    \begin{equation}
        y(t) = -2e^{-t}u(t) + 3e^{-2t}u(t)
    \end{equation}

    \includegraphics[]{1c.pdf}
    \item %write the solution of q1d

\begin{equation}
    \frac{dy}{dt} + y =  \frac{dx}{dt} - x
\end{equation}

\begin{equation}
    x + y =  \frac{dx}{dt} -  \frac{dy}{dt}
\end{equation}
\begin{equation}
   \int (x + y)  dt = \int (\frac{dx}{dt} -  \frac{dy}{dt}) dt
\end{equation}
\begin{equation}
   \int x dt  + \int y dt = x -  y
\end{equation}
\begin{equation}
    y = x -  \int x dt  - \int y dt
\end{equation}

\tikzset{every picture/.style={line width=0.75pt}} %set default line width to 0.75pt        

\begin{tikzpicture}[x=0.75pt,y=0.75pt,yscale=-1,xscale=1]
%uncomment if require: \path (0,300); %set diagram left start at 0, and has height of 300

%Straight Lines [id:da7726647444431063] 
\draw    (73,102) -- (343,102) ;
\draw [shift={(346,102)}, rotate = 180] [fill={rgb, 255:red, 0; green, 0; blue, 0 }  ][line width=0.08]  [draw opacity=0] (8.93,-4.29) -- (0,0) -- (8.93,4.29) -- cycle    ;
%Straight Lines [id:da1155812830674563] 
\draw    (164,102) -- (163.39,179.86) ;
%Straight Lines [id:da1536587320856233] 
\draw    (163.39,179.75) -- (247.85,179) ;
%Shape: Ellipse [id:dp24004365179803977] 
\draw   (247.85,179) .. controls (247.85,162.92) and (259.53,149.89) .. (273.93,149.89) .. controls (288.33,149.89) and (300,162.92) .. (300,179) .. controls (300,195.08) and (288.33,208.11) .. (273.93,208.11) .. controls (259.53,208.11) and (247.85,195.08) .. (247.85,179) -- cycle ;
%Shape: Ellipse [id:dp905012926935391] 
\draw   (346,102) .. controls (346,85.92) and (362.79,72.89) .. (383.5,72.89) .. controls (404.21,72.89) and (421,85.92) .. (421,102) .. controls (421,118.08) and (404.21,131.11) .. (383.5,131.11) .. controls (362.79,131.11) and (346,118.08) .. (346,102) -- cycle ;
%Straight Lines [id:da28303836446124286] 
\draw    (300,179) -- (375,180) ;
%Straight Lines [id:da3761212524701436] 
\draw    (375,180) -- (374.52,133.11) ;
\draw [shift={(374.5,131.11)}, rotate = 89.41] [color={rgb, 255:red, 0; green, 0; blue, 0 }  ][line width=0.75]    (10.93,-3.29) .. controls (6.95,-1.4) and (3.31,-0.3) .. (0,0) .. controls (3.31,0.3) and (6.95,1.4) .. (10.93,3.29)   ;
%Straight Lines [id:da07565415017997634] 
\draw    (421,102) -- (589,102.99) ;
\draw [shift={(591,103)}, rotate = 180.34] [color={rgb, 255:red, 0; green, 0; blue, 0 }  ][line width=0.75]    (10.93,-3.29) .. controls (6.95,-1.4) and (3.31,-0.3) .. (0,0) .. controls (3.31,0.3) and (6.95,1.4) .. (10.93,3.29)   ;
%Straight Lines [id:da6555384616591895] 
\draw    (506,102.5) -- (508,177) ;
%Shape: Ellipse [id:dp960544376811661] 
\draw   (431.85,177.94) .. controls (431.85,160.24) and (443.53,145.89) .. (457.93,145.89) .. controls (472.33,145.89) and (484,160.24) .. (484,177.94) .. controls (484,195.65) and (472.33,210) .. (457.93,210) .. controls (443.53,210) and (431.85,195.65) .. (431.85,177.94) -- cycle ;
%Straight Lines [id:da08859920180216796] 
\draw    (384,179) -- (383.52,133.11) ;
\draw [shift={(383.5,131.11)}, rotate = 89.4] [color={rgb, 255:red, 0; green, 0; blue, 0 }  ][line width=0.75]    (10.93,-3.29) .. controls (6.95,-1.4) and (3.31,-0.3) .. (0,0) .. controls (3.31,0.3) and (6.95,1.4) .. (10.93,3.29)   ;
%Straight Lines [id:da2699624755617147] 
\draw    (384,179) -- (431.85,177.94) ;
%Straight Lines [id:da8849118252100503] 
\draw    (484,177.94) -- (508,177) ;

% Text Node
\draw (264,162.4) node [anchor=north west][inner sep=0.75pt]  [font=\LARGE]  {$\int $};
% Text Node
\draw (374,87.4) node [anchor=north west][inner sep=0.75pt]  [font=\Large]  {$+$};
% Text Node
\draw (447,160.4) node [anchor=north west][inner sep=0.75pt]  [font=\LARGE]  {$\int $};
% Text Node
\draw (39,93.4) node [anchor=north west][inner sep=0.75pt]    {$x( t)$};
% Text Node
\draw (600,94.4) node [anchor=north west][inner sep=0.75pt]    {$y( t)$};
% Text Node
\draw (358,139.4) node [anchor=north west][inner sep=0.75pt]    {$-$};
% Text Node
\draw (391,139.4) node [anchor=north west][inner sep=0.75pt]    {$-$};


\end{tikzpicture}
    \end{enumerate}
\item %write the solution of q2  
	\begin{enumerate}
    % Write your solutions in the following items.
    \item %write the solution of q2a
    \begin{equation}
        H(e^{jw}) = \frac{Y(e^{jw})}{X(e^{jw})}
    \end{equation}
    \begin{equation}
        e^{jw}Y(e^{jw}) -\frac{1}{2}Y(e^{jw}) = e^{jw}X(e^{jw})
    \end{equation}
    \begin{equation}
        H(e^{jw}) = \frac{Y(e^{jw})}{X(e^{jw})} = \frac{e^{jw}}{e^{jw}-\frac{1}{2}} = \frac{1}{1-\frac{1}{2}e^{-jw}}
    \end{equation}
    \item %write the solution of q2b
    \begin{equation}
        h[n] = F^{-1}\{H(e^{jw})\} = F^{-1}\{\frac{1}{1-\frac{1}{2}e^{-jw}}\}  = (\frac{1}{2})^nu[n]
    \end{equation}
	\item %write the solution of q2c
    \begin{equation}
        F^{-1}\{X(e^{jw})\} = \frac{1}{1 - \frac{3}{4}e^{-jw}}
    \end{equation}
    \begin{equation}
        y[n] = x[n] * h[n]
    \end{equation}
    \begin{equation}
        Y(e^{jw}) = X(e^{jw})H(e^{jw})
    \end{equation}
    \begin{equation}
        Y(e^{jw}) = \frac{1}{1 - \frac{3}{4}e^{-jw}}\frac{1}{1 - \frac{1}{2}e^{-jw}} = \frac{A}{1 - \frac{3}{4}e^{-jw}} + \frac{B}{1 - \frac{1}{2}e^{-jw}}
    \end{equation}
    \begin{equation}
        A + B = 1, \frac{1}{2}A + \frac{3}{4}B = 0
    \end{equation}
    \begin{equation}
        A = 3, B = -2
    \end{equation}
    \begin{equation}
        Y(e^{jw}) = \frac{3}{1 - \frac{3}{4}e^{-jw}} - \frac{2}{1 - \frac{1}{2}e^{-jw}}
    \end{equation}
    \begin{equation}
        y[n] = 3(\frac{3}{4})^nu[n] - 2(\frac{1}{2})^nu[n]
    \end{equation}
    
    \end{enumerate}

\item %write the solution of q3
	\begin{enumerate}
    % Write your solutions in the following items.
    \item %write the solution of q3a
    \begin{equation}
        H_3(jw) = H_2(jw)H_1(jw) = \frac{1}{jw^2 + 3jw + 2}
    \end{equation}
    \begin{equation}
        H_3(jw) = \frac{Y(jw)}{X(jw)} = \frac{1}{jw^2 + 3jw + 2}
    \end{equation}
    \begin{equation}
        jw^2Y(jw) + 3jwY(jw) + 2Y(jw) = X(jw)
    \end{equation}
    \begin{equation}
        y''(t) + 3y'(t) + 2y(t) = x(t)
    \end{equation}
    
    \item %write the solution of q3b
    
    \begin{equation}
        H_3(jw) = H_2(jw)H_1(jw) = \frac{1}{(jw+1)(jw+2)} = \frac{A}{jw+1} + \frac{B}{jw+2}
    \end{equation}
    \begin{equation}
        A = 1, B= -1
    \end{equation}
    \begin{equation}
        h(t) = F^{-1}\{H(jw)\} = e^{-t}u(t) - e^{-2t}u(t)
    \end{equation}
    
	\item %write the solution of q3c
    \begin{equation}
        y(t) = x(t)*h(t) 
    \end{equation}
    \begin{equation}
        Y(jw) = X(jw)H_3(jw)
    \end{equation}
    \begin{equation}
        Y(jw) = \frac{jw}{(jw+1)(jw+2)} = \frac{A}{jw+1} + \frac{B}{jw+2}
    \end{equation}
    \begin{equation}
        2A+B=0, A+B=1
    \end{equation}
    \begin{equation}
        A=-1, B=2
    \end{equation}
    \begin{equation}
        Y(jw) = \frac{-1}{jw+1} + \frac{2}{jw+2}
    \end{equation}
    \begin{equation}
        y(t) = -e^{-t}u(t) + 2e^{-2t}u(t)
    \end{equation}
    \end{enumerate}

\item %write the solution of q4
    \begin{enumerate}   
    % Write your solutions in the following items.
    \item %write the solution of q4a
    \begin{equation}
        y[n] = x[n]*h_1[n] + x[n]*h_2[n]
    \end{equation}
    \begin{equation}
        y[n] = x[n]*(h_1[n]+h_2[n])
    \end{equation}
    \begin{equation}
        Y(e^{jw}) = X(e^{jw})(H_1(e^{jw})+H_2(e^{jw}))
    \end{equation}
    \begin{equation}
        Y(e^{jw}) = X(e^{jw})H(e^{jw})        
    \end{equation}
    \begin{equation}
        H(e^{jw}) = H_1(e^{jw})+H_2(e^{jw})
    \end{equation}
    \begin{equation}
        H(e^{jw}) = \frac{ 5e^{-jw} + 12}{e^{-2jw} + 5e^{-jw} + 6}
    \end{equation}
    \begin{equation}
        H(e^{jw}) = \frac{Y(e^{jw})}{X(e^{jw})}
    \end{equation}
    \begin{equation}
        Y(e^{jw})(e^{-2jw} + 5e^{-jw} + 6) = (5e^{-jw} + 12)X(e^{jw})
    \end{equation}
    \begin{equation}
        y[n-2] + 5y[n-1] + 6y[n] = 5x[n-1] + 12x[n]
    \end{equation}
    \item %write the solution of q4b
    \begin{equation}
        H(e^{jw}) = H_1(e^{jw})+H_2(e^{jw})
    \end{equation}
    \begin{equation}
        H(e^{jw}) = \frac{ 5e^{-jw} + 12}{e^{-2jw} + 5e^{-jw} + 6}
    \end{equation}
	\item %write the solution of q4c
    \begin{equation}
        h[n] = F^{-1}\{H(e^{jw})\} = F^{-1}\{\frac{3}{3+e^{-jw}} + \frac{2}{2+e^{-jw}}\} =  F^{-1}\{\frac{1}{1+\frac{1}{3}e^{-jw}} + \frac{1}{1+\frac{1}{2}e^{-jw}}\}
    \end{equation}
    \begin{equation}
        h[n] = (-\frac{1}{3})^nu[n] + (-\frac{1}{2})^nu[n]
    \end{equation}
    \end{enumerate}

\item %write the solution of q5
The code including the implementation of FFT and IFFT algorithms and are below:

\begin{lstlisting}[language=Python, caption=Message decoder that uses FFT\&IFFT implementaton ]
import numpy as np
import matplotlib.pyplot as plt
import scipy.io.wavfile as wavfile

def fft(x):
    N = int(len(x))

    if N <= 1:
        return x
    
    even = fft(x[::2])
    odd = fft(x[1::2])


    X = np.zeros(N, dtype=np.complex128)
    for k in range(int(N/2)):
        factor = np.exp(-2j * np.pi * (k-1) / N)
        X[k] = odd[k] + factor * even[k]
        if k == 0:
            X[k + int (N/2)] = odd[k] + np.exp(-2j * np.pi * (int (N/2) -1) / N) * even [k]
        else:    
            X[k + int (N/2)] = odd[k] - factor * even [k]

    return X


def ifft(x):
    inverse = fft(np.conj(x))/len(x)
    return np.real(np.conj(inverse))
    

if __name__ == '__main__':

    # read encoded.wav
    sample_rate, data = wavfile.read('encoded.wav')

    print('sample rate: ', sample_rate)

    duration = len(data) / sample_rate
    time = np.linspace(0, duration, len(data))

    # plot the encoded signal in time domain
    fig1 = plt.figure()
    plt.plot(time,data)
    plt.title('Encoded Signal in Time Domain')
    plt.xlabel('Time(s)')
    plt.ylabel('Amplitude')
    plt.show()
    fig1.savefig('encoded_time.pdf')



    frequency_domain = fft(data)
    frequencies = np.linspace(-len(data)//2, len(data)//2, len(data))

    # plot the magnitude of the frequency domain
    fig2 = plt.figure()
    plt.plot(frequencies,np.abs(frequency_domain))
    plt.title('Encoded Signal in Frequency Domain')
    plt.xlabel('Frequency(Hz)')
    plt.ylabel('Amplitude')
    plt.show()
    #save as pdf
    fig2.savefig('encoded_freq.pdf')

    modified_freq = np.concatenate((np.flip(frequency_domain[:len(frequency_domain)//2]), np.flip(frequency_domain[len(frequency_domain)//2:]))) #solves the secret

    # plot the magnitude of the modified frequency domain
    fig3 = plt.figure()
    plt.plot(frequencies,np.abs(modified_freq))
    plt.title('Decoded Signal in Frequency Domain')
    plt.xlabel('Frequency()')
    plt.ylabel('Amplitude')
    plt.show()
    fig3.savefig('decoded_freq.pdf')

    modified_time = ifft(modified_freq)

    # plot the decoded signal in time domain
    fig4 = plt.figure()
    plt.plot(time, modified_time)
    plt.title('Decoded Signal in Time Domain')
    plt.xlabel('Time(s)')
    plt.ylabel('Amplitude')
    plt.show()
    fig4.savefig('decoded_time.pdf')

    # write the modified time domain to a wav file
    wavfile.write('decoded.wav', sample_rate, modified_time.real.astype(data.dtype))






\end{lstlisting}

\textbf{Plots:}

\begin{figure}[!htb]
    \centering
    \includegraphics{encoded_time.pdf}
    \caption{Encoded Signal in Time Domain}
    \label{fig:enter-label}
\end{figure}
\begin{figure}[!htb]
    \centering
    \includegraphics{encoded_freq.pdf}
    \caption{Encoded Signal in Frequency Domain}
    \label{fig:enter-label}
\end{figure}
\begin{figure}[!htb]
    \centering
    \includegraphics{decoded_freq.pdf}
    \caption{Decoded Signal in Frequency Domain}
    \label{fig:enter-label}
\end{figure}
\begin{figure}[!htb]
    \centering
    \includegraphics{decoded_time.pdf}
    \caption{Decoded Signal in Time Domain}
    \label{fig:enter-label}
\end{figure}
\newpage
The secret message is:
\begin{center}
\Huge
\textbf{\textit{"I have a dream."}}
\end{center}

\end{enumerate}


\end{document}

